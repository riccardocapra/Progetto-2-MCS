\documentclass[a4paper,12pt,titlepage,oneside]{book}
\linespread{1.2}
\pagestyle{plain}


\usepackage[italian]{babel} 
\usepackage{picture}
\usepackage{hyperref}
\usepackage[Symbol]{upgreek}
\usepackage{amsmath}
\usepackage{wrapfig}
\usepackage{siunitx}
\usepackage{caption}
\usepackage{subcaption}
\usepackage{graphicx}
\usepackage{csquotes}
\usepackage{color}
\usepackage{listings}
\usepackage{xcolor}

\definecolor{codegreen}{rgb}{0,0.6,0}
\definecolor{codegray}{rgb}{0.5,0.5,0.5}
\definecolor{codepurple}{rgb}{0.58,0,0.82}
\definecolor{backcolour}{rgb}{0.95,0.95,0.92}

\lstdefinestyle{mystyle}{
    backgroundcolor=\color{backcolour},   
    commentstyle=\color{codegreen},
    keywordstyle=\color{magenta},
    numberstyle=\tiny\color{codegray},
    stringstyle=\color{codepurple},
    basicstyle=\ttfamily\footnotesize,
    breakatwhitespace=false,         
    breaklines=true,                 
    captionpos=b,                    
    keepspaces=true,                 
    numbers=left,                    
    numbersep=5pt,                  
    showspaces=false,                
    showstringspaces=false,
    showtabs=false,                  
    tabsize=2
}

\lstset{style=mystyle}

\usepackage{pdfpages}
\textheight24cm\topmargin0mm\headheight0mm\headsep6mm\oddsidemargin20pt\evensidemargin30pt

\usepackage[T1]{fontenc}
\usepackage[utf8]{inputenc}
\usepackage{titlesec, blindtext, color}
\definecolor{gray75}{gray}{0.75}
\newcommand{\hsp}{\hspace{20pt}}
\titleformat{\chapter}[hang]{\Huge\bfseries}{\thechapter\hsp\textcolor{gray75}{|}\hsp}{0pt}{\Huge\bfseries}

\usepackage{csquotes}
\usepackage[backend=bibtex]{biblatex}
\bibliography{bibliografia}


\definecolor{dkgreen}{rgb}{0,0.6,0}
\definecolor{gray}{rgb}{0.5,0.5,0.5}
\definecolor{mauve}{rgb}{0.58,0,0.82}

\begin{document}

\begin{titlepage}

\newcommand{\HRule}{\rule{\linewidth}{0.5mm}} % Defines a new command for the horizontal lines, change thickness here

\center % Center everything on the page
 
%----------------------------------------------------------------------------------------
%	HEADING SECTIONS
%----------------------------------------------------------------------------------------

\textsc{\LARGE Università degli studi di Milano-Bicocca}\\[1cm] % Name of your university/college
\textsc{\Large Metodi del Calcolo Scientifico }\\[0.3cm] % Major heading such as course name
\textsc{\large Project 2}\\[0.1cm] % Minor heading such as course title

%----------------------------------------------------------------------------------------
%	TITLE SECTION
%----------------------------------------------------------------------------------------

\HRule \\[0.4cm]
{ \huge \bfseries Compressione di immagini tramite la
DCT}\\[0.4cm] % Title of your document
\HRule \\[1.5cm]
 
%----------------------------------------------------------------------------------------
%	AUTHOR SECTION
%----------------------------------------------------------------------------------------

\large
\emph{Authors:}\\
Capra Riccardo - 808227 \\ % Your name
Curnis Giovanni - 807424 \\[1cm]

% If you don't want a supervisor, uncomment the two lines below and remove the section above
%\Large \emph{Author:}\\
%John \textsc{Smith}\\[3cm] % Your name

%----------------------------------------------------------------------------------------
%	DATE SECTION
%----------------------------------------------------------------------------------------

%{\large \today}\\[1cm] % Date, change the \today to a set date if you want to be precise

%----------------------------------------------------------------------------------------
%	LOGO SECTION
%----------------------------------------------------------------------------------------

\includegraphics[scale = 2.5]{src/image/unimib-logo.pdf}\\[1cm] % Include a department/university logo - this will require the graphicx package
 
%----------------------------------------------------------------------------------------

\vfill % Fill the rest of the page with whitespace

\end{titlepage}

\tableofcontents

\chapter{Introduzione}
Questo progetto è diviso in due fasi: della prima fase è implementare un algoritmo di $DCT_2$ in ambiente Open Source e confrontare questa implementazione con quella di una libreria ottimizzata, scopo della seconda fase è creare un interfaccia che consenta di comprimere immagini in toni di grigio per immagini in formato BMP.

\chapter{Cenni teorici}

Per codificare una immagine in scala di grigi o a colori l'operazione che viene effettuata la maggior parte delle volte è quella della trasformata che consente di rappresentare i pixel, normalmente rappresentati da una matrice contenete i valori di luminosità di ogni pixel, in una rappresentazione tramite frequenze. In questo tipo di rappresentazione le onde a bassa frequenza rendono conto dell'andamento generale della luminosità dell'immagine, mentre quelle ad alta frequenza codificano i dettagli e il disturbo introdotto dall'elettronica dei dispositivi di acquisizione.

Normalmente le immagini fotografiche hanno un andamento continuo e quindi la rappresentazione nel dominio delle frequenze  è spesso costituita da una matrice quadrata di valori, dei quali solo quelli corrispondenti alle basse frequenze sono significativi e si trovano nell'angolo in alto a sinistra mentre le frequenze alte hanno una rilevanza minore.

Attenuare ulteriormente le componenti meno significative è possibile, basta dividere ogni coefficiente della matrice trasformata per un certo valore specificato in una tabella, così facendo la matrice risultante conterrà numerosi valori nulli. Questa operazione però causa una grande perdita di qualità ed è nota come quantizzazione e viene usata per ottenere una elevata compressione dell'immagine.

\section{$DCT_2$}
I gruppi di standardizzazione JPEG e MPEG utilizzano la \textbf{Trasformata Discreta del Coseno (DCT)} che essendo capace di rivelare le variazioni di informazione tra un’area e quella contigua trascurando le ripetizioni provvede alla compressione spaziale. Inoltre, essa rappresenta un ottimo compromesso complessità-prestazioni.

Tuttavia la complessità dell'algoritmo DCT cresce molto rapidamente al crescere delle dimensioni dell'immagine e per questa ragione la DCT viene applicata su blocchi più piccoli, ad esempio di 8x8 \textit{px} nei quali viene scomposta.

I numeri contenuti in questi blocchi 8x8 vengono detti \textbf{coefficienti DCT} e descrivono il contenuto delle frequenze sia alte che basse in quel blocco. In teoria questa trasformata dovrebbe essere fedele ma le operazioni in virgola mobile coinvolte e le varie operazioni inducono degli errori di rappresentazione e causano delle perdite di informazione.

Una conseguenza vantaggiosa della DCT è che i valori significativi saranno trasportati in alto a sinistra.

Questo ragionamento è facilmente applicabile anche a più dimensioni. Prendendo un considerazione un array il cui indice è bidimensionale otterremo la seguente formula:
\begin{equation}
    c_{kl} = \alpha_{Kl} \sum^{N-1}_{i=0} \sum^{M-1}_{j=0} f_{ij} \cos ( k \pi \frac{2i + 1}{2N} ) \cos (l \pi \frac{2j + 1}{2M})
\end{equation}

\begin{itemize}
  \item \textbf{$c_kl$} è il coefficiente che il segnale \textbf{f} assume nel dominio delle frequenze del segnale con frequenza  orizzontale \textbf{k} e frequenza verticale \textbf{l}.
  \item \textbf{$\alpha_{kl}$}  è il fattore di normalizzazione per rendere le basi (sia componente orizzontale sia componente verticale) ortonormali.
  \item $f_{ij}$ è il segnale in due dimensioni (l’immagine), di dimensioni(N,M), valutato in in (i, j).
  \item i 2 \textbf{\cos} sono i componenti delle basi rispettivamente rispetto
alle righe e alle colonne di frequenza \textbf{k} e \textbf{l}.
  
\end{itemize}


\section{$IDCT_2$}
La DCT non crea nessuna compressione. Se un decoder (per esempio il player MPEG) esegue l’operazione inversa, la IDCT, si riottiene il blocco di, per esempio, 8x8 pixel senza aver guadagnato nulla. Per memorizzare la luminosità degli 8x8 pixel occorrono 64 byte e in ugual numero ne servono per memorizzare i 64 coefficienti della DCT.

Anche la IDCT è estendibile a più dimensioni ottenendo la seguente formula:

\begin{equation}
    f_{ij} = \sum^{N-1}_{k=0} \sum^{M-1}_{l=0} c_{kl} \alpha_{kl} \cos ( k \pi \frac{2i + 1}{2N} ) \cos ( l \pi \frac{2J + 1}{2M} )
\end{equation}

\begin{itemize}

\item $\alpha_{00} = \frac{1}{\sqrt{NM}}$
\item $\alpha_{k0} = \alpha_{01} = \sqrt{\frac{2}{NM}}$
\item $\alpha_{kl} = \frac{2}{\sqrt{NM}}, k, l \geq 1$

\end{itemize}

\chapter{Impostazione del Progetto}
    \section{Nostra implementazione DCT}
    Di seguito il codice utilizzato per implementare la DCT.
    \lstinputlisting[language=Python,label={pythoncode}, caption={Codice Python $DCT \And IDCT$}]{src/code/homeMadeDCT.py}
    \section{Chiamata FFT}
    Di seguito lo snippet utilizzato per chiamare la funzione ottimizzata FFT
    \lstinputlisting[language=Python,label={pythoncode}, caption={Codice Python chiamata FFT}]{src/code/FFT.py}
    
    \section{Codice confronto}
    Per effettuare un confronto tra i due programmi e misurarne le differenze.

    
    \section{Librerie Utilizzate}
        
    
    \section{Ambiente di Benchmark}
    La macchina utilizzata per risolvere i sitemi è un \textbf{Satellite L50-A-1D6} con le seguenti caratteristiche:
    \begin{itemize}
      \item \textbf{Processore}: i7-4700MQ
      \item \textbf{RAM}: 8 GB DDR3
      \item \textbf{HDD}: 1TB SSD
      \item \textbf{Versione Windows}: Window 10 Pro
      \item \textbf{Versione Ubuntu}: Ubuntu 18.04
    \end{itemize}
    
\chapter{Analisi dei Risultati}
    \section{Cofronto MATLAB tra OS}
    

\chapter{Conclusioni}

\nocite{*}
\printbibliography

\end{document}